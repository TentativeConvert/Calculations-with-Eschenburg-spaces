% -*- coding: utf-8-unix -*-
\documentclass{article}
\usepackage[bottom=2.5cm, top=2.5cm, left=4.5cm, right=4.5cm]{geometry}
\usepackage{fontspec} 
\usepackage{polyglossia}
    \setmainlanguage[spelling=new]{british}
\usepackage{microtype,booktabs,xcolor,paralist}
\usepackage{mathtools,amsthm,amsrefs}
\mathtoolsset{mathic}

\swapnumbers
\newtheorem{thm}{Theorem}
\newtheorem{lem}[thm]{Lemma}
\newtheorem{prop}[thm]{Proposition}
\newtheorem{eg}[thm]{Example}
\newtheorem{claim}[thm]{Claim}


\newcommand{\abs}[1]{\left|#1\right|}

\usepackage{tabularx}
\newcolumntype{L}{>{\raggedright\arraybackslash}X}
\newcolumntype{R}{>{\raggedleft\arraybackslash}X}
\newcolumntype{C}{>{\centering\arraybackslash}X}

\begin{document}
\section*{Limits}
In the following we consider an Eschenburg space \(E\) with parameters \(k_1,k_2,k_3\), \(l_1,l_2,l_3\).  Write \(\abs{r} := |H^4(E)|\) and \(s_2 := s_2(E)\).

\begin{claim}
  If \(\abs{r} \leq R\) and \(\abs{k_i},\abs{l_i} \leq P\), then in order for the computations of the invariants to be reliable, the data types used need to meet the following requirements:
  \begin{center}
    \begin{tabular}{rp{20em}}
      \verb+INT_R+ & (signed) integer with capacity of at least ... bits \\
      \verb+INT_P+ & (signed) integer with capacity of at least ... bits \\
      \verb+INT_KS+ & (signed) integer with capacity of at least ... bits \\
      \verb+FLOAT_KS+ & float with significand of at least ... bits
    \end{tabular}
  \end{center}
  These requirements are sufficient provided the implementation of the sin function  \verb+boost/math/special_functions/sin_pi.hpp+
  is as exact as the data type \verb+FLOAT_KS+ permits.
\end{claim}

\begin{eg}
  The default data types specified in \verb+config.h+ are:
  \begin{center}
    \begin{tabular}{rp{20em}}
      \verb+INT_R+ & \texttt{:= long} (at east $..$ bit) \\
      \verb+INT_P+ & \texttt{:= long} (at least $..$ bit) \\
      \verb+INT_KS+ & \texttt{:= long long} (at least $..$ bit) \\
      \verb+FLOAT_KS+ & \texttt{:= double} ($53$ bit significand)
    \end{tabular}
  \end{center}
  The implementation of \verb+sin_pi+ for \texttt{double} using the GNU C++ compiler is exact\footnote{%
    \url{http://www.boost.org/doc/libs/1_65_1/libs/math/doc/html/math_toolkit/powers/sin_pi.html}}.
  Thus, by the above claim, computations are reliable in the following ranges:
  \begin{compactitem}
  \item For analysing a single space with parameters \(\abs{k_i}, \abs{l_i} \leq ...\).

  \item  For generating list of spaces in standard parametrization with \(\abs{r} \leq ...\).    
  \end{compactitem}
\end{eg}


In the following we consider an Eschenburg space \(E\) with parameters \(k_1,k_2,k_3\), \(l_1,l_2,l_3\).  Write \(\abs{r} := |H^4(E)|\) and \(s_2 := s_2(E)\).
The invariant \(s_2\) is computed by a formula of the form
\begin{align}\label{eq:s2}
  \notag  s_2   &= (q-2)/d + \ell_1 + \ell_2 + \ell_3\\
                &= \frac{45(q-2) + [45\ell_1] + [45\ell_2] + [45\ell_3]}{45d}
\end{align}  
where \(\ell_i\) are lens space invariants such that \(45\ell_i\) is an integer.  

In the following proposition, \(R\) and \(P\) need to be assumed sufficiently large.
\begin{prop}
  Suppose \(E\) is a space in standard presentation with \(\abs{r}\leq R\).
  Then denominator and numerator of \(\abs{s_2}\) are bounded by 
  \(
      2^{17,1}·R^{5/2}.
  \)
  The absolute values of the integers \(d\), \(q\) and \(45\ell_i\) appearing in \eqref{eq:s2} are bounded by the same value.

  Suppose \(E\) is an Eschenburg space such that the absolute values of all paremeters are bounded by \(P\).  Then the denominator and the numerator of \(\abs{s_2}\) are bounded by 
  \(
    2^{16,7}·P^{5}.
  \)
\end{prop}

{\color{red}
  What does ``mantissa'' mean?  Should I make separate statements for the float values \(45\ell_i\)?

  What does sufficiently large mean?
}


\begin{table}[b]
  \begin{center}
  \begin{tabular}{rrrrr} 
    \toprule
    data type  & bits    & \text{range}    & standard \(R\)  & general \(P\)   \\
    \midrule                               
    int / long & \(32\)  & \(\pm 2^{31}\)  & \(47\)          & \(7\)           \\ 
    long long  & \(64\)  & \(\pm 2^{63}\)  & \(336\,442\)    & \(613\)         \\
    ??         & \(128\) & \(\pm 2^{127}\) & \(1,7·10^{13}\) & \(4\,372\,418\) \\
    \bottomrule
  \end{tabular}
  % 
  % log_2(y) = ln(y)/ln(2)
  % 
  % 2^{17,1}·R^{5/2} <=  2^x
  % <=>                R <= 2^{2/5 (x-17,1)}
  % 
  % 2^{16,7}·P^5     <=  2^x 
  % <=>                P <= 2^{1/5 (x-16,7)}
  % 
  \caption{Different data types and the resulting bounds \(R\) on \(\abs{r}\) (for spaces in standard presentation) and \(P\) on the parameters (for any presentation), according to the above proposition.  (The values in the first line of the table may not actually be ``sufficiently large'' for the proposition to apply.)}
\end{center}
\end{table}

\begin{lem}\label{lem:s2}
  Suppose the absolute values of \(\abs{k_i}\) and \(\abs{l_i}\) are bounded by \(P\),
  and \(\abs{r}\) is bounded by \(R\). Then  the denominator and the numerator of \(\abs{s_2}\) are bounded as follows:
  \begin{align*}
    \abs{\text{numerator}}   &\leq 2·3^3·5·P^4\\
    \abs{\text{denominator}} &\leq 2^7·3^3·5· R P^3
  \end{align*}
  (provided \(P\) and \(R\) are sufficiently large).

  All intermediate integer variables used in the computation of \(s_2\) in [...] are bounded by the same value. 
\end{lem}
 
\begin{lem}\label{lem:P-vs-R}
  For any Eschenburg space, 
  \begin{align*}
    \abs{k_i}, \abs{l_i} \leq P &\quad\Rightarrow\quad \abs{r} \leq 6 P^2\\
    \intertext{%
    For an Eschenburg space in standard presentation,}
    \abs{k_i}, \abs{l_i} \leq 2R^{1/2} &\quad\Leftarrow\quad \abs{r} \leq R 
  \end{align*}
\end{lem}

\begin{proof}[Proof of Lemma~\ref{lem:s2}]
The invariant \(s_2\) is computed by a formula of the form
\begin{align*}
  s_2   &= (q-2)/d + \ell_1 + \ell_2 + \ell_3\\
        &= \frac{45(q-2) + [45\ell_1] + [45\ell_2] + [45\ell_3]}{45d}
\end{align*}  
where \(\ell_i\) are lens space invariants such that \(45\ell_i\) is an integer.  
The absolute value of \(q\) in this formula is bounded by a sum of six squares of differences of parameters \((k_i-l_j)\), so 
\begin{align}
  \abs{q}&\leq 6(2P)^2 \leq 2^3·3·P^2.\\
  \intertext{
  The absolute value of \(d\) is bounded by}
  \notag 
  \abs{d} &\leq 3 · 2^4 · R · (2P)^3\\
         &\leq 2^7·3·R P^3
\end{align}
Each lens invariant \(\ell_i\) is a sum of \(p\) summands of the form
\[
  \abs{cos(…)-1}
  ·\abs{1/sin\left(\frac{k\pi p_1}{p}\right)}
  ·\abs{1/sin\left(\frac{k\pi p_2}{p}\right)}
  ·\abs{1/sin\left(\frac{k\pi p_3}{p}\right)}
  ·\abs{1/sin\left(\frac{k\pi p_4}{p}\right)}
\]
When \(\abs{x}\) is small, \(sin(x) \sim x\), so an upper bound for such a summand can be estimated as
\begin{align*}
   \abs{cos(…)-1}
    ·\abs{\frac{p}{k\pi p_1}}
    ·\abs{\frac{p}{k\pi p_2}}
    ·\abs{\frac{p}{k\pi p_3}}
    ·\abs{\frac{p}{k\pi p_4}}
 \leq 2·\frac{\abs{p}^4}{\pi^4k^4}
  \leq 2^{-5} \frac{p^4}{k^4} 
\end{align*}
Summing over \(k\), we obtain:
\begin{align*}
  \abs{\ell_i} 
  &\leq 2^{-5}\abs{p}^4 \sum_{k=1}^{\abs{p}}\left(\frac{1}{k^4}\right)\\
  &\leq 2^{-4}\abs{p}^4.
\end{align*}
Finally, \(\abs{p}\leq 2P\)  because the parameter \(p\) is a difference of two parameters of \(E\), so
\begin{equation}
\abs{\ell_i} \leq P^4 .
\end{equation}
Thus, altogether we obtain the following bounds for numerator and denominator of \(s_2\):
\begin{align*}
  \abs{\text{numerator}} &\leq 45·(3·2^3·P^2 + 3·P^4) \leq 2·3^3·5·P^4\\
  \abs{\text{denominator}} &\leq 45·\abs{d} \leq 2^7·3^3·5· R P^3
\end{align*}
\end{proof}

\begin{proof}[Proof of Lemma~\ref{lem:P-vs-R}]
The first implication is clear from \(r = \sigma_2(k_1,k_2,k_3) - \sigma_2(l_1,l_2,l_3)\).
For the second implication, note that all parameters execpt \(k_3\) are bounded by \(\sqrt{R}\) in the standard representation; the parameter \(k_3\) is bounded only by \(2\sqrt{R}\).
\end{proof}


\begin{proof}[Proof of the proposition]
The proposition is immediate from the two lemmas and the estimates of upper bounds for the values of \(q\), \(d\) and \(45\ell_i\) appearing in the proof of Lemma~\ref{lem:s2}.  In both cases, it is clear that for sufficiently large \(R\) and \(P\) the bound for the denominator of \(\abs{s_2}\) is the largest bound that occurs.  For Eschenburg spaces in standard presentation, this bound is 
\(
2^{10}·3^3·R^{5/2} \leq 2^{17,1}·R^{5/2}.
\)
For general Eschenburg spaces, this bound is
\(
    2^{8}·3^4·5·P^{5} \leq 2^{16,7}·P^5
\).
\end{proof}
\end{document}

%%% Local Variables: 
%%% coding: utf-8
%%% mode: latex
%%% TeX-engine: xetex
%%% End: 