% -*- coding: utf-8-unix -*-
\documentclass{article}
\usepackage[bottom=2.5cm, top=2.5cm, left=4.5cm, right=4.5cm]{geometry}
\usepackage{fontspec} 
\usepackage{polyglossia}
    \setmainlanguage[spelling=new]{british}
\usepackage{microtype,booktabs,paralist,nicefrac,multicol}
\usepackage{mathtools,amsthm}
\usepackage[alphabetic,initials]{amsrefs}
\mathtoolsset{mathic}
\usepackage[dvipsnames]{xcolor}
\usepackage{hyperref}
\hypersetup{breaklinks=true,colorlinks=true,linkcolor=black,anchorcolor=black,citecolor=black,filecolor=black,menucolor=black,urlcolor=MidnightBlue}
\urlstyle{same}
\usepackage{fancyvrb}
\VerbatimFootnotes

%\swapnumbers
%\newtheorem{thm}{Theorem}
\newtheorem{lem}{Lemma}
\newtheorem{prop}{Proposition}
\newtheorem{note}{Note}
%\newtheorem{eg}[thm]{Example}
\newtheorem{claim}{Claim}


\newcommand{\abs}[1]{\left|#1\right|}
\renewcommand{\vec}[1]{\mathbf{#1}}
\newcommand{\macheps}{\epsilon} % machine epsilon

\newcommand{\expP}{\mathcal E_{\text{P}}}
\newcommand{\expR}{\mathcal E_{\text{R}}}
\newcommand{\expKS}{\mathcal E_{\text{KS}}}
\newcommand{\sigKS}{\mathcal S_{\text{KS}}}

\usepackage{tabularx}
\newcolumntype{L}{>{\raggedright\arraybackslash}X}
\newcolumntype{R}{>{\raggedleft\arraybackslash}X}
\newcolumntype{C}{>{\centering\arraybackslash}X}

\setlength{\parindent}{0pt}
\addtolength{\parskip}{3pt}

\begin{document}
\section*{Limits}
Consider an Eschenburg space \(E\) with parameters \((k_1,k_2,k_3,l_1,l_2,l_3)\).   We say that the \textbf{parameters are bounded by \(P\)} if \(\abs{k_i}\leq P\) and \(\abs{l_i}\leq P\) for all~\(i\) for some positive integer \(P\).  Similarly, we say that \textbf{\(r\) is bounded by \(R\)} if \(\abs{r(E)}\leq R\) for some positive integer~\(R\), where  \(\abs{r(E)} = |H^4(E)|\).  We say that \(E\) is given in \textbf{standard presentation} if the parameters satisfy the conditions of \cite{CEZ}*{Lemma~1.4}.  (All spaces generated using \verb+esch -r=XXXX+ are in standard presentation.)

The default data types specified in \verb+config.h+ and their sizes are:\footnote{
  \mbox{\url{http://www.boost.org/doc/libs/1_65_1/libs/multiprecision/doc/html/boost_multiprecision/tut/ints/cpp_int.html}}%
}

  \begin{tabular}{rp{22em}l}
    \verb+INT_P+ & \verb+:= std::int_least32_t+ & \(\geq 32\) bit \\
    \verb+INT_R+ & \verb+:= std::int_least64_t+ & \(\geq 64\) bit \\
    \verb+INT_KS+ & \verb+:= boost::multiprecision::int128_t+ & \(\geq 128\) bit \\
    \verb+FLOAT_KS+ & \verb+:= long double+ & \(64\) bit significand
  \end{tabular}

The implementation of the \(\sin\) function used by \verb+esch+\footnote{
  We use \verb+sin_pi+ from \verb+boost/math/special_functions/sin_pi.hpp+.%
}
has a relative error of less than \(1\macheps\)\footnote{\label{footnote:macheps}%
  Here, \(\macheps\) denotes the \textbf{machine epsilon}.
  See \url{https://en.wikipedia.org/wiki/Machine_epsilon}.%
}
for the data type \texttt{long double}.\footnote{
  \mbox{\url{http://www.boost.org/doc/libs/1_65_1/libs/math/doc/html/math_toolkit/powers/sin_pi.html}}%
}

\begin{claim}\label{claim:default}
  With the above configurations, the output of the program \verb+esch+ is reliable in the following ranges:
  \begin{compactitem}
  \item For computing the invariants of an arbitrary space with parameters bounded by \(P=1500\).
  \item For generating and analysing list of spaces in standard parametrization with \(r\) bounded by \(R=600\,000\).
  \end{compactitem}
\end{claim}

More generally, we claim the following:
\begin{claim}\label{claim:bits}
  Suppose \(r\) is bounded by \(R\) and the parameters are bounded by \(P\).
  Suppose further that the data types used meet the following minimum requirements:
  
  \begin{tabular}{rp{\linewidth-5em}}
    \verb+INT_R+ & signed integer of size \(\expR\) bits \\
    \verb+INT_P+ & signed integer of size \(\expP\) bits \\
    \verb+INT_KS+ & signed integer of size \(\expKS\) bits \\
    \verb+FLOAT_KS+ & base-\(2\) float with significand of \(\sigKS\) bits (including sign bit)
  \end{tabular}
  
  Suppose further that, in the computation of the invariant \(s_{2}(E)\), the \(\sin\)-values are computed with a relative error of at most \(A\macheps\).\footnote{
    See footnote \ref{footnote:macheps}.
  }
  Then the computations of the invariants \(r(E)\), \(s(E)\), \(p_1(E)\), \(s_2(E)\) and \(s_{22}(E)\)
  are exact provided each of the following inequalities is satisfied:
  \begin{multicols}{2}
    \noindent\allowdisplaybreaks
  \begin{align*}
    P & \leq 2^{\expP-1}  \label{eq:bits:PeP}\tag{$a$}          \\
    R    & \leq 2^{\expR-1}  \label{eq:bits:ReR}\tag{$b$}      \\
    RP^8 & \leq 2^{\expKS-15.3} \label{eq:bits:PeKS} \tag{$c$}   \\
    P^5  & \leq 2^{\sigKS-10.5}\label{eq:bits:PsKS}·A^{-1} \tag{$d$}
  \end{align*}
  \end{multicols}
\end{claim}

To obtain Claim~\ref{claim:default} from Claim~\ref{claim:bits}, we will use that the bounds \(P\) and \(R\) can be related as follows:
\begin{note}\label{lem:P-vs-R}
  For any Eschenburg space, 
  \begin{alignat*}{7}
    &\text{parameters bounded by } P &&\quad\Rightarrow\quad r \text{ bounded by } R = 6 P^2\\
    \intertext{%
    For an Eschenburg space in standard presentation,}
    &\text{parameters bounded by } P = 2R^{1/2} &&\quad\Leftarrow\quad r \text{ bounded by } R 
  \end{alignat*}
\end{note}

\subsection*{Verification of Claim~\ref{claim:default} using Claim~\ref{claim:bits}}

We first verify Note~\ref{lem:P-vs-R}.
The first implication is clear from \(r = \sigma_2(k_1,k_2,k_3) - \sigma_2(l_1,l_2,l_3)\).
For the second implication, note that while all parameters except \(k_3\) are bounded by \(\sqrt{R}\) in the standard presentation, the parameter \(k_3\) is bounded only by \(2\sqrt{R}\).

To find the bound for \(P\) when analysing a single space, we can replace \(R\) by \(6P^2\) in all inequalities in Claim~\ref{claim:bits}. With \(\expP=32\), \(\expR=64\) and \(\expKS=128\) and \(A=1\) the standard values specified above, these inequalities become:
\begin{multicols}{2}
  \noindent\allowdisplaybreaks
  \begin{align*}
    P & \leq 2^{31}  \label{eq:bits:PeP}\tag{$a$}     \\
    6P^2 & \leq 2^{63}  \label{eq:bits:ReR}\tag{$b$}        \\
    6P^{10} & \leq 2^{112.7} \label{eq:bits:RPeKS} \tag{$c$} \\
    P^5  & \leq 2^{53.5}\label{eq:bits:PsKS} \tag{$d$}
  \end{align*}
\end{multicols}
Here, the strongest inequality is inequality \((d)\), which equates to \(P\leq 1663\).

To find a bound for \(R\) when analysing spaces in standard presentation, repace \(P\) by \(2\sqrt{R}\) in all inequalities in Claim~\ref{claim:bits}.  They become:
\begin{multicols}{2}
  \noindent\allowdisplaybreaks
  \begin{align*}
    2\sqrt{R} & \leq 2^{31}  \label{eq:bits:PeP}\tag{$a$}     \\
    R    & \leq 2^{63}  \label{eq:bits:ReR}\tag{$a'$}        \\
    2^8·R^5  & \leq 2^{112.7} \label{eq:bits:PeKS} \tag{$b$}  \\
    2^5\sqrt{R}^5  & \leq 2^{53.5}\label{eq:bits:PsKS} \tag{$d$}
  \end{align*}
\end{multicols}
Again, the strongest inequality is inequality \((d)\).  It equates to \(R\leq 691\,802\).

\subsection*{Preliminary inequalities I (for integer types)}
The invariants \(s_2(E)\) and \(s_{22}(E)\) are computed by formulas of the following form \cite{CEZ}*{(2.1)}:
\begin{align}\label{eq:s2}
  \notag  s_2(E)   &= (q-2)/d + \ell_1 + \ell_2 + \ell_3\\
                   &= \frac{45(q-2) + ([45\ell_1] + [45\ell_2] + [45\ell_3])d}{45d}\\
  s_{22}(E) &= 2 \abs{r(E)} s_2(E) 
\end{align}  
Here, \(\ell_i\) are lens space invariants such that \(45\ell_i\) is an integer \cite{CEZ}*{Prop.~3.13}.  

\begin{prop}\label{prop:s2-integers}
  Suppose the parameters are bounded by \(P\) and \(r\) is bounded by \(R\).  Then the absolute values of the denominators and the numerators of \(s_2(E)\) and \(s_{22}(E)\), and the absolute values of the integers \(d\), \(q\) and \(45\ell_i\) appearing in \eqref{eq:s2}, are bounded by \(2^{15.3}RP^8\).
\end{prop}
 
\begin{proof}
The integer \(d\) in \eqref{eq:s2} \(=\) \cite{CEZ}*{(2.1)} is a multiple of \(2r(E)\).  Thus, any bounds for numerator and denominator of \(s_2(E)\) will also be bounds for numerator and denominator of \(s_{22}(E)\).

The absolute value of \(q\) in \eqref{eq:s2} is bounded by a sum of six squares of differences of parameters \((k_i-l_j)\), so 
\begin{alignat}{7}
  \abs{q}&\leq 6(2P)^2 &&< 2^{4.6}·P^2\\
  \intertext{
  The absolute value of \(d\) is bounded by}
  \abs{d} &\leq 3 · 2^4 · R · (2P)^3 &&< 2^{8.6} · RP^3
\end{alignat}
An upper bound for the values of  \(45\ell_i\) is estimated as \(2^{5.1}P^5\) in  Propsition~\ref{prop:45ell} below. 
Thus, altogether we obtain the following bounds for numerator and denominator of \(s_2\):
\begin{align*}
  \abs{\text{numerator}} &\leq 45·2^{4.6}·P^2 + 3·2^{5.1}P^5·2^{8.6}RP^3 \approx 2^{15.3}·RP^8\\
  \abs{\text{denominator}} &\leq 45·\abs{d} \leq 2^{14.1}· R P^3
\end{align*}
Clearly, the first bound is greater than the second. 
\end{proof}

\subsection*{Preliminary inequalities II (for the float type)}
The lens invariants \(\ell_1\), \(\ell_2\), \(\ell_3\) are computed as a sum
\begin{equation}\label{eq:ell}
  \ell_i := \sum_{v=1}^{\abs{p}-1} L(\vec x^{(v)}),
\end{equation}
where each \(\vec x^{(v)}=(x_0^{(v)},\dots,x_4^{(v)})\) is a quintuple of real numbers and the function \(L\) is given by
\begin{equation}\label{eq:L}
  L(\vec x) 
  := (\cos(\pi x_0)-1)
  · \textstyle\prod_{i=1}^{4} \csc(\pi x_i),
\end{equation}
where \(\csc(x) := 1/\sin(x)\).
The coordinates \(x_i\) on which this function is evaluated are given by \(x_0^{(v)} = \frac{v}{p}\) and \(x_i^{(v)} = \frac{vp_i}{p}\) for \(i=1,\dots, 4\), with each of \(p\), \(p_1\), \dots, \(p_4\) a difference of parameters \(k_i-l_j\).  

\begin{prop}\label{prop:45ell}
  If the parameters of the given Eschenburg space are bounded by \(P\), then \(\abs{45\ell_i} \leq 2^{5.1} P^5\) for each \(i\in\{1,2,3\}\).
\end{prop}
\begin{proof}
  As each of \(k_i\) and \(l_i\) is bounded by \(P\), each of the parameters \(p\), \(p_1\), \dots, \(p_4\) used to define the quintuples \(\vec x^{(v)}\) is bounded by \(2P\).  It follows that each coordinate of each \(\vec x^{(v)}\) has a distance of at least \(\nicefrac{1}{2P}\) to the nearest integer.  Thus, we can apply Lemma~\ref{lem:csc} below to each \(\csc\)-factor of \(L(\vec x^{(v)})\) with \(\varepsilon = \nicefrac{1}{2P}\) to obtain 
  \[
    \abs{L(\vec x^{(v)})}
    \leq 2·(2^{-1.6}·2P)^4 = 2^{-1.4}P^4\\
  \]
  Now take a \((\abs{p}-1)\)-fold sum and multiply by~\(45\), and note that \((\abs{p}-1) < 2P\) and  \(45 < 2^{5.5}\).
\end{proof}

\begin{lem}\label{lem:csc}
  Let \(\varepsilon>0\) be sufficiently small (\(\leq \nicefrac{1}{100}\)).  Then
  \(\abs{\csc(\pi·x)} \leq 2^{-1.6} \varepsilon^{-1}\) for any real number \(x\) whose distance to the nearest integer is at least~\(\varepsilon\).
\end{lem}
\begin{proof}
  It suffices to show that \(\sin(\pi·x) \geq 2^{1.6} \varepsilon\) for any real \(x\in [\varepsilon,\nicefrac{1}{2}]\), where \(\varepsilon \in (0,\nicefrac{1}{100})\) is some given lower bound.  As \(\tan(\pi x) \geq \pi x\) for all \(x\in [0,\nicefrac{1}{2})\), we have \(\sin(\pi x) \geq \pi x·\cos(\pi x)\) for all \(x\in [0,\nicefrac{1}{2}]\).  So for \(x\in [\varepsilon,\nicefrac{1}{2}]\) we find that 
  \[
    \sin(\pi x) \geq \sin(\pi\varepsilon) \geq \pi\varepsilon·\cos(\pi\varepsilon).
  \]
  If \(\varepsilon\) is sufficiently small, then \(\pi·\cos(\pi\varepsilon)\) is close to \(\pi\).
  The result is obtained by explicitly calculating this value for \(\varepsilon = \nicefrac{1}{100}\).
% cos(\pi/10) = 2^{−0,072397019}
% \pi         = 2^{ 1,651496129}
%
% So \pi (cos(\pi/10)) > 2^{1,6} 
%
\end{proof}

\subsection*{Verification of claim~\ref{claim:bits}}
The inequalities \((a)\) and \((b)\) simply state that \verb+INT_P+ and \verb+INT_K+ need to be large enough to hold the values of the parameters and the value of \(r(E)\), respectively.  The data type \verb+INT_KS+ must be large enough to hold all integers used in computing \(s_2(E)\), so inequality \((c)\) follows directly from Proposition~\ref{prop:s2-integers}.  

It remains to verify that the data type \verb+FLOAT_KS+ is sufficiently precise to compute the integer values \(45\ell_i\) appearing in~\eqref{eq:s2}.  By assumption, the \(\sin\)-values in \eqref{eq:ell}\slash\eqref{eq:L} are computed with a relative error of at most \(\eta = A\macheps\), where \(\macheps=2^{1-\sigKS}\).  (Note that one bit of the significand is used to store the sign of the number, so we only have \(\sigKS-1\) bits to store the value.) That is, the computed values of \(\sin(\pi x_i)\) may differ from the actual values by a factor of at most \(1\pm \delta\). As the coordinates \(x_i^{(v)}\) used as input to the \(\sin\)-functions may also be exact only up to a factor of \(1\pm\delta\), we find that altogether the computed values of \(\sin(\pi x_i)\) may differ from the actual values by a factor of \((1\pm \delta)^2\), and likewise for the values of \(\cos(\pi x_0)\). It follows that the computed values of \(L\) and \(\ell_i\) differ from the actual values by a factor of \((1\pm\delta)^2(1\mp\delta)^{-10} \approx 1 \pm 10\delta < 1+A·2^{4.4-\sigKS}\).  By Proposition~\ref{prop:45ell}, this leads to an absolute error for \(45\ell_i\) of at most \[
  A·2^{4.4-\sigKS}·\abs{45\ell_i} = A·2^{4.4-\sigKS}·2^{5.1}P^5 = A·2^{9.5-\sigKS}·P^5.
\]
To obtain the correct integer value of \(45\ell_i\) after rounding, we need this absolute error to be less than \(0.5 = 2^{-1}\).  That is, we need
\(
A·2^{9.5-\sigKS}·P^5 < 2^{-1},
\)
or, equivalently,
\(
P^5 < A^{-1}·2^{\sigKS-10.5}.
\)
This is inequality~\((d)\).



\subsection*{Reference}
\renewcommand*{\MR}[1]{\href{http://www.ams.org/mathscinet-getitem?mr=#1}{{\sffamily ~~[MR#1]}}}
\begin{biblist}
  \bib{CEZ}{article}{
    author={Chinburg, Ted},
    author={Escher, Christine},
    author={Ziller, Wolfgang},
    title={Topological properties of Eschenburg spaces and 3-Sasakian
      manifolds},
    journal={Math. Ann.},
    volume={339},
    date={2007},
    number={1},
    pages={3--20},
    issn={0025-5831},
    review={\MR{2317760}},
  }
\end{biblist}

\end{document}

%%% Local Variables: 
%%% coding: utf-8
%%% mode: latex
%%% TeX-engine: xetex
%%% End: 

