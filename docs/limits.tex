% -*- coding: utf-8-unix -*-
\documentclass{article}
\usepackage[bottom=2.5cm, top=2.5cm, left=4.2cm, right=4.2cm]{geometry}
\usepackage{fontspec} 
\usepackage{polyglossia}
    \setmainlanguage[spelling=new]{british}
\usepackage{microtype,booktabs,paralist,nicefrac,multicol}
\usepackage{mathtools,amsthm,amsrefs}
\mathtoolsset{mathic}
\usepackage[dvipsnames]{xcolor}
\usepackage{hyperref}
\hypersetup{breaklinks=true,colorlinks=true,linkcolor=black,anchorcolor=black,citecolor=black,filecolor=black,menucolor=black,urlcolor=MidnightBlue}
\urlstyle{same}


%\swapnumbers
%\newtheorem{thm}{Theorem}
\newtheorem{lem}{Lemma}
\newtheorem{prop}{Proposition}
\newtheorem{note}{Note}
%\newtheorem{eg}[thm]{Example}
\newtheorem{claim}{Claim}


\newcommand{\abs}[1]{\left|#1\right|}
\renewcommand{\vec}[1]{\mathbf{#1}}
\newcommand{\macheps}{\epsilon} % machine epsilon

\newcommand{\expP}{\mathcal E_{\text{P}}}
\newcommand{\expR}{\mathcal E_{\text{R}}}
\newcommand{\expKS}{\mathcal E_{\text{KS}}}
\newcommand{\sigKS}{\mathcal S_{\text{KS}}}

\usepackage{tabularx}
\newcolumntype{L}{>{\raggedright\arraybackslash}X}
\newcolumntype{R}{>{\raggedleft\arraybackslash}X}
\newcolumntype{C}{>{\centering\arraybackslash}X}

\setlength{\parindent}{0pt}
\addtolength{\parskip}{3pt}

\begin{document}
\section*{Limits}
Consider an Eschenburg space \(E\) with parameters \((\vec k,\vec l) = (k_1,k_2,k_3,l_1,l_2,l_3)\).   We say that the \textbf{parameters are bounded by \(P\)} for some positive integer \(P\) if \(\abs{k_i}\leq P\) and \(\abs{l_i}\leq P\) for all~\(i\).  Similarly, we say that \textbf{\(r\) is bounded by \(R\)} if \(\abs{r(E)}\leq R\) for some positive integer~\(R\), where  \(\abs{r(E)} = |H^4(E)|\).  

The default data types specified in \verb+config.h+ and their sizes on my system are:
\begin{center}
  \begin{tabular}{rp{10em}l}
    \verb+INT_R+ & \texttt{:= int} & \(32\) bit \\
    \verb+INT_P+ & \texttt{:= long} &  \(64\) bit \\
    \verb+INT_KS+ & \texttt{:= long long} & \(64\) bit \\
    \verb+FLOAT_KS+ & \texttt{:= long double} & \(64\) bit significand
  \end{tabular}
\end{center}
The implementation of the sin function \verb+boost/math/special_functions/sin_pi.hpp+
for the data type \texttt{long double} has a relative error of less than \(1\macheps\).\footnote{
  \mbox{\url{http://www.boost.org/doc/libs/1_65_1/libs/math/doc/html/math_toolkit/powers/sin_pi.html}}}

\begin{claim}\label{claim:default}
  With the above configurations, the output of the program \verb+esch+ is reliable in the following ranges:
  \begin{compactitem}
  \item For analysing a single space with parameters bounded by \(P=140\).
  \item For generating and analysing list of spaces in standard parametrization with \(r\) bounded by \(R=300\,000\).
  \end{compactitem}
\end{claim}

More generally, we claim the following:
\begin{claim}\label{claim:bits}
  Suppose \(r\) is bounded by \(R\) and the parameters are bounded by \(P\).
  Suppose further that the data types used meet the following minimum requirements:
  
  \begin{tabular}{rp{\linewidth-5em}}
    \verb+INT_R+ & signed integer of size \(\expR\) bits \\
    \verb+INT_P+ & signed integer of size \(\expP\) bits \\
    \verb+INT_KS+ & signed integer of size \(\expKS\) bits \\
    \verb+FLOAT_KS+ & base-\(2\) float with significand of \(\sigKS\) bits (including sign bit)
  \end{tabular}
  
  Suppose further that, in the of the computation of the invariant \(s_{2}(E)\), the \(\sin\)-values are computed with a relative error of at most \(A\macheps\).\footnote{
    Here, \(\macheps\) denotes the \textbf{machine epsilon}.
    See \url{https://en.wikipedia.org/wiki/Machine_epsilon}.
  }
  Then the computations of the invariants \(r(E)\), \(s(E)\), \(p_1(E)\) and \(s_2(E)\)
  are exact provided each of the following inequalities is satisfied:
  \begin{multicols}{3}
    \noindent\allowdisplaybreaks
  \begin{align*}
    P & \leq 2^{\expP-1}  \label{eq:bits:PeP}\tag{$a$}     \\
    R    & \leq 2^{\expR-1}  \label{eq:bits:ReR}\tag{$a'$}        \\
    P^5  & \leq 2^{\expKS-7.7} \label{eq:bits:PeKS} \tag{$b$}  \\
    RP^3 & \leq 2^{\expKS-15.1} \label{eq:bits:RPeKS} \tag{$b'$} \\
    P^5  & \leq (2/A)^{\sigKS-9.5}\label{eq:bits:PsKS} \tag{$c$}
  \end{align*}
  \end{multicols}
\end{claim}

To obtain Claim~\ref{claim:default} from Claim~\ref{claim:bits}, we will use that the bounds \(P\) and \(R\) are related as follows:
\begin{note}\label{lem:P-vs-R}
  For any Eschenburg space, 
  \begin{alignat*}{7}
    &\text{parameters bounded by } P &&\quad\Rightarrow\quad r \text{ bounded by } R = 6 P^2\\
    \intertext{%
    For an Eschenburg space in standard presentation,}
    &\text{parameters bounded by } P = 2R^{1/2} &&\quad\Leftarrow\quad r \text{ bounded by } R 
  \end{alignat*}
\end{note}

\subsection*{Verification of claim~\ref{claim:default} using claim~\ref{claim:bits}}

We first verify Note~\ref{lem:P-vs-R}.
The first implication is clear from \(r = \sigma_2(k_1,k_2,k_3) - \sigma_2(l_1,l_2,l_3)\).
For the second implication, note that while all parameters except \(k_3\) are bounded by \(\sqrt{R}\) in the standard representation, the parameter \(k_3\) is bounded only by \(2\sqrt{R}\).

Now, to find the bound for \(P\) when analysing a single space, we can replace \(R\) by \(6P^2\) in all inequalities in Claim~\ref{claim:bits}. With \(\expP=32\), \(\expR=64\) and \(\expKS=64\) and \(A=1\) the standard values specified above, these inequalities become:
\begin{multicols}{3}
  \noindent\allowdisplaybreaks
  \begin{align*}
    P & \leq 2^{31}  \label{eq:bits:PeP}\tag{$a$}     \\
    6P^2 & \leq 2^{63}  \label{eq:bits:ReR}\tag{$a'$}        \\
    P^5  & \leq 2^{56.3} \label{eq:bits:PeKS} \tag{$b$}  \\
    6P^5 & \leq 2^{48.9} \label{eq:bits:RPeKS} \tag{$b'$} \\
    P^5  & \leq 2^{54.5}\label{eq:bits:PsKS} \tag{$c$}
  \end{align*}
\end{multicols}
Here, the strongest inequality is inequality \((2')\), which equates to \(P\leq 146\).

To find a bound for \(R\) when analysing spaces in standard presentation, repace \(P\) by \(2\sqrt{R}\) in all inequalities in Claim~\ref{claim:bits}.  They become:
\begin{multicols}{3}
  \noindent\allowdisplaybreaks
  \begin{align*}
    2\sqrt{R} & \leq 2^{31}  \label{eq:bits:PeP}\tag{$a$}     \\
    R    & \leq 2^{63}  \label{eq:bits:ReR}\tag{$a'$}        \\
    2^5·\sqrt{R}^5  & \leq 2^{56.3} \label{eq:bits:PeKS} \tag{$b$}  \\
    2^3 \sqrt{R}^5 & \leq 2^{48.9} \label{eq:bits:RPeKS} \tag{$b'$} \\
    2^5\sqrt{R}^5  & \leq 2^{54.5}\label{eq:bits:PsKS} \tag{$c$}
  \end{align*}
\end{multicols}
Again, the strongest inequality is inequality \((2')\).  It equates to \(R\leq 336\,442\).

\subsection*{Preliminary inequalities I (for integer types)}
The invariant \(s_2\) is computed by a formula of the form
\begin{align}\label{eq:s2}
  \notag  s_2   &= (q-2)/d + \ell_1 + \ell_2 + \ell_3\\
                &= \frac{45(q-2) + [45\ell_1] + [45\ell_2] + [45\ell_3]}{45d}
\end{align}  
where \(\ell_i\) are lens space invariants such that \(45\ell_i\) is an integer.  

\begin{prop}\label{prop:s2-integers}
  Suppose the parameters are bounded by \(P\) and \(r\) is bounded by \(R\).  Then the denominator and the numerator of \(\abs{s_2}\) are bounded as follows:
  \begin{align*}
    \abs{\text{numerator}}   &\leq 2^{6.7}·P^5\\
    \abs{\text{denominator}} &\leq 2^{14.1}· RP^3
  \end{align*}
  The absolute values of the integers \(d\), \(q\) and \(45\ell_i\) appearing in \eqref{eq:s2} are bounded by the same value as the denominator of \(s_2\).
\end{prop}
 
\begin{proof}
The absolute value of \(q\) in \eqref{eq:s2} is bounded by a sum of six squares of differences of parameters \((k_i-l_j)\), so 
\begin{alignat}{7}
  \abs{q}&\leq 6(2P)^2 &&< 2^{4.6}·P^2\\
  \intertext{
  The absolute value of \(d\) is bounded by}
  \notag 
  \abs{d} &\leq 3 · 2^4 · R · (2P)^3 &&< 2^{8.6} · RP^3
\end{alignat}
An upper bound for the values of  \(45\ell_i\) is estimated as \(2^{5.1}P^5\) in  Propsition~\ref{prop:45ell} below. 
Thus, altogether we obtain the following bounds for numerator and denominator of \(s_2\):
\begin{align*}
  \abs{\text{numerator}} &\leq 45·2^{4.6}·P^2 + 3·2^{5.1}P^5 \approx 2^{6.7}·P^5\\
  \abs{\text{denominator}} &\leq 45·\abs{d} \leq 2^{14.1}· R P^3\qedhere
\end{align*}
\end{proof}

\begin{proof}[Proof of the proposition]
The proposition is immediate from the two lemmas and the estimates of upper bounds for the values of \(q\), \(d\) and \(45\ell_i\) appearing in the proof of Lemma~\ref{prop:s2-integers}.  In both cases, it is clear that for sufficiently large \(R\) and \(P\) the bound for the denominator of \(\abs{s_2}\) is the largest bound that occurs.  For Eschenburg spaces in standard presentation, this bound is 
\(
2^{10}·3^3·R^{5/2} \leq 2^{17,1}·R^{5/2}.
\)
For general Eschenburg spaces, this bound is
\(
    2^{8}·3^4·5·P^{5} \leq 2^{16,7}·P^5
\).
\end{proof}

\subsection*{Preliminary inequalities II (for the float type)}
The lens invariants \(\ell_1\), \(\ell_2\), \(\ell_3\) are computed as a sum
\begin{equation}\label{eq:ell}
  \ell_i := \sum_{v=1}^{\abs{p_0}-1} L(\vec x^{(v)}),
\end{equation}
where each \(\vec x^{(v)}=(x_0^{(v)},\dots,x_4^{(v)})\) is a quintuple of real numbers and the function \(L\) is given by
\begin{equation}\label{eq:L}
  L(\vec x) 
  := (\cos(\pi x_0)-1)
  · \textstyle\prod_{i=1}^{4} \csc(\pi x_i),
\end{equation}
where \(\csc(x) := 1/\sin(x)\).
The coordinates \(x_i\) on which this function is evaluated are given by \(x_i^{(v)} = \frac{v p_i}{p}\) with each of \(p_0\), \(p_1\), \dots, \(p_4\) a difference of parameters \(k_i-l_j\).  

\begin{prop}\label{prop:45ell}
  If the parameters are bounded by  \(P\), then \(\abs{45\ell_i} \leq 2^{5.1} P^5\) for each \(i\in\{1,2,3\}\).
\end{prop}
\begin{proof}
  As each of \(k_i\) and \(l_i\) is bounded by \(P\), each of the parameters \(p\), \(p_1\), \dots, \(p_4\) used to define the quintuples \(\vec x^{(v)}\) is bounded by \(2P\).  It follows that each coordinate of each \(\vec x^{(v)}\) has a distance of at least \(\nicefrac{1}{2P}\) to the nearest integer.  Thus, we can apply Lemma~\ref{lem:csc} below to each \(L(\vec x^{(v)})\) with \(\epsilon = 1/(2P)\) to obtain 
  \[
    \abs{L(\vec x^{(v)})}
    \leq 2^{-5.4}(\nicefrac{1}{2P})^{-4} = 2^{-1.4}P^4\\
  \]
  Now take a \((\abs{p}-1)\)-fold sum and multiply by~\(45\), and note that \((\abs{p}-1) < 2P\) and  \(45 < 2^{5.5}\).
\end{proof}

\begin{lem}\label{lem:csc}
  Let \(\varepsilon>0\) be sufficiently small (\(\leq \nicefrac{1}{100}\)).  Then
  \(\abs{\csc(\pi·x)} \leq 2^{-1.6} \varepsilon^{-1}\) for any real number \(x\) whose distance to the nearest integer is at least~\(\varepsilon\).
\end{lem}
\begin{proof}
  It suffices to show that \(\sin(\pi·x) \geq 2^{1.6} \varepsilon\) for any real \(x\in [\varepsilon,\nicefrac{1}{2}]\), where \(\varepsilon \in (0,\nicefrac{1}{100})\) is some given lower bound.  It is known that \(\sin(\pi x) \geq \pi x·\cos(\pi x)\) for all \(x\in [0,\nicefrac{1}{2}]\), so for \(x\in [\varepsilon,\nicefrac{1}{2}]\) we find that 
  \[
    \sin(\pi x) \geq \sin(\pi\varepsilon) \geq \pi\varepsilon·\cos(\pi\varepsilon).
  \]
  If \(\varepsilon\) is sufficiently small, then \(\pi·\cos(\pi\varepsilon)\) is close to \(\pi\).
  The result is obtained by explicitly calculating this value for \(\varepsilon = \nicefrac{1}{100}\).
% cos(\pi/10) = 2^{−0,072397019}
% \pi         = 2^{ 1,651496129}
%
% So \pi (cos(\pi/10)) > 2^{1,6} 
%
\end{proof}

\subsection*{Verification of claim~\ref{claim:bits}}
The inequalities \((a)\), \((a')\), \((b)\) and \((b')\) follow directly from Proposition~\ref{prop:s2-integers}.  

For inequality \((c)\), we first note that by assumption the \(\sin\)-values in \eqref{eq:ell}\slash\eqref{eq:L} are computed with a relative error of at most \(\eta = A\macheps\), where \(\macheps=2^{1-\sigKS}\).  (Note that one bit of the significand is used to store the sign of the number, so we only have \(\sigKS-1\) bits to store the value.) That is, the computed value of \(\sin(\dots)\) differs from the actual value by a factor of at most \(1\pm \delta\).  It follows that the computed values of \(L\) and \(\ell_i\) differ from the acutal values by a factor of approximately \((1+\delta)(1-\delta)^{-4} \approx 1 + 5\delta < 1+A·2^{3.4-\sigKS}\).  By Proposition~\ref{prop:45ell}, this leads to an absolute error of approximately \(A·2^{3.4-\sigKS}·2^{5.1}P^5 = A·2^{8.5-\sigKS}\).  We know that the exact value of \(45\ell_i\) is an integer.  To obtain the correct integer, we need this absolute error to be less than \(0.5 = 2^{-1}\).  That is, we need
  \(
    A·2^{8.5-\sigKS}·P^5 < 2^{-1},
  \)
  or, equivalently,
  \(
    P^5 < A^{-1}·2^{\sigKS-9.5}.
  \)
  This implies inequality~\((c)\).
\end{document}

%%% Local Variables: 
%%% coding: utf-8
%%% mode: latex
%%% TeX-engine: xetex
%%% End: 